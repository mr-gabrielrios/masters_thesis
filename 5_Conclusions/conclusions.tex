\chapter{Conclusions and future work}
\label{chapter:Conclusions}
\thispagestyle{myheadings}

\graphicspath{{4_Conclusion/Figures/}}

This thesis provides insights into the capabilities of remote sensing for improving observations such that our understanding of the UBL can be improved. In this chapter, results from the work done to establish these capabilities and the associated findings are summarized in Section \ref{section:summary}. The limitations of the methods used will also be discussed in Section \ref{section:limitations}, as there are many improvements that can be made upon the methods used herein. In addition to the limitations and possible improvements, the potential directions the work performed could be taken in to build upon this topic will be discussed in \ref{section:future_work}.

\section{Summary}\label{section:summary}

This thesis contributes novel findings regarding UBL structure, dynamics, and energy exchange with a focus on using remote sensing methods to drive analysis of the UBL. This begins with Chapter \ref{chapter:goes}, where a lightweight algorithm is presented that uses GOES-R satellite data and a \SI{2}{\meter} air temperature machine learning method to estimate $Q_H$ in urban areas. This work is relevant to urban meteorology and remote sensing of surface processes, as it presents a new and portable method to leverage operational satellite data for real-time estimation of surface fluxes. The satellite-derived estimates of $Q_H$ were validated by comparison to several flux towers in New York City, with a resulting $R^2$ value of 0.70 and an MBE of \SI{16.58}{\watt\per\meter\squared}. This work resulted in a peer-reviewed journal publication \citep{rios2022novel} and has shown to outperform a dedicated urban numerical weather prediction model with regards to the estimation of $Q_H$. Additionally, this work demonstrated the spatial and temporal variability of $Q_H$ within New York City. In terms of spatial variability, higher values of $Q_H$ were presented in highly-urbanized neighborhoods with minimal vegetation, indicating that $Q_H$ is responsible for the majority of the urban surface energy balance and that higher values of $Q_S$ are present in these areas. In terms of temporal variability, a seasonal cycle of $Q_H$ was apparent with spring and summer months exhibiting the strongest surface fluxes. 

To supplement the usage of satellite-derived data to improve our understanding of UBL processes, the work performed in Chapter \ref{chapter:climatology} used ground-based remote sensing methods to vertically-profile the UBL to analyze UBL structure and dynamics with an emphasis on UBL behavior during extreme heat events. In this chapter, lidar and microwave radiometers, situated in 4 boroughs and operating for 3 years continuously, are used to provide long-term observations of the UBL over New York City. The number of stations, coupled with a maximum vertical resolution of \SI{100}{\meter}, allows for a comprehensive spatial analysis of the UBL in a large and diverse coastal urban area. Particular focus was given to extreme heat events, which present a significant natural hazard for New York City. Findings from observations and the successive analysis indicate that extreme heat events present a site-averaged \SI{2}{\meter} air temperature increase of \SI{7}{K} for daily high temperatures, with a 39.4\% increase in specific humidity and a southwesterly shift in winds throughout the entirety of the UBL. Moreover, extreme heat events present a statistically-significant increase to temperature and moisture throughout the vertical extent of the UBL. It was also observed that sea breezes from the Atlantic Ocean and Long Island Sound contributed to reduced afternoon and evening temperatures and were critical for onshore moisture transport during nighttime hours, although these effects were directly proportional to the distance of the observation site from these bodies of water. Findings from this work have resulted in a submission to a peer-reviewed journal and are aimed to improve the understanding of how extreme heat events affect conditions both at the surface, but also throughout the UBL, as a proper understanding of the UBL is critical for improving modeling and forecasting methods.

\section{Limitations}\label{section:limitations}

Various limitations affected the work performed herein. Methods to address these limitations are discussed in Section \ref{section:future_work}. 

With regards to the satellite-driven algorithm for $Q_H$ estimation presented in Chapter \ref{chapter:goes}, the heterogeneity of urban surfaces is a parameter that presents shortcomings to propertly estimating surface fluxes throughout an area with land cover surface properties as complex and variable as in New York City. The usage of satellite data with \SI{2}{\kilo\meter} resolution, while more than sufficient for many application, is limited for areas with such granular surface properties. The inability to capture surface properties at lower resolutions using this satellite dataset was identified as a key contributor to estimation errors during the validation process for this product. Additionally, the complex flow structure presented by the urban canopy in New York City creates a highly heterogeneous flow environment that can lead to intra-neighborhood variability in surface fluxes at a resolution that is not captured by the model. 

With regards to the observations and analysis using ground-based remote sensing methods presented in Chapter \ref{chapter:climatology}, a key shortcoming is the fact that each observational site performed is a single point, geographically speaking. In an environment as complex as New York City, a point is insufficient to capture the microscale turbulence that occurs as a function of differential building heights and surface cover types that can change within a matter of meters. Although the usage of sites from several boroughs attempts to provide some sort of spatial variability, the results presented may vary to some degree from one block to the next. This is especially true for observations made within the roughness sublayer, as mixing is driven by turbulent processes arising from surface properties.

\section{Future Work}\label{section:future_work}

The fields of urban meteorology and boundary layer research are constantly evolving and will be increasing in societal relevance due to the increasing degree of urbanization globally and the effects of climate change. Given this, the work presented here can be expanded upon and be a stepping stone to much greater findings with social and technical impacts.

The work presented in Chapter \ref{chapter:goes} can readily be applied to derive surface latent heat fluxes ($Q_L$) using satellite data. The critical step to adapt the algorithm for $Q_H$ towards estimating $Q_L$ involves the incorporation of a method to estimate evaporation and transporation in urban areas. These factors will be functions of vegetation, soil moisture, and bodies of water within urban areas that contribute to vertical moisture transport and horizontal advection. Additionally, the application of both of these methods could prove useful for validation of numerical weather prediction models, as the algorithm adds a spatial extent that builds upon the point-based observation data provided by flux towers that is currently used for validation. This validation method may be able to assist research into improvements for these models, especially in urban areas, where numerical models may fail to capture local-scale atmospheric phenomena. Upon completion of the algorithm for $Q_L$, a complete satellite-based notable step forward in identifying the individual components of the surface energy budget. This has crucial implications for fields such as urban planning and public health, as mitigation and placement strategies can be informed by understanding the factors that contribute to surface energy budget components and their relationships to heat and human comfort. 

The work presented in Chapter \ref{chapter:climatology} is currently being expanded upon to focus on UBL dynamics. The focus on UBL dynamics is important because of the work needed to better understand turbulence throughout the UBL. A specific emphasis is being placed on the relationship between the surface and mixed layers, as their interface (the roughness sublayer) has been found to be a layer within which MOST fails to capture turbulent processes. The analysis of turbulence in the UBL is being performed with several high-resolution lidars to capture a wide range of turbulent processes, with the aim of using methods such as spectral analysis and statistical filters to identify turbulent eddies, mixing timescales, and the differences between turbulent properties at different levels in the UBL. 

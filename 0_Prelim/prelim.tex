% This file contains all the necessary setup and commands to create
% the preliminary pages according to the buthesis.sty option.

\title{Understanding the relationship between urban areas and the boundary layer using remote sensing methods}

\author{Gabriel Angel Rios}

\department{Department of Mechanical Engineering}

% Degree year is the year the diploma is expected, and defense year is
% the year the dissertation is written up and defended. Often, these
% will be the same, except for January graduation, when your defense
% will be in the fall of year X, and your graduation will be in
% January of year X+1
\degreeyear{2022}

% For each reader, specify appropriate label {First, Second, Third},
% then name, and title. IMPORTANT: The title should be:
%   "Professor of Electrical and Computer Engineering",
% or similar, but it MUST NOT be:
%   Professor, Department of Electrical and Computer Engineering"
% or you will be asked to reprint and get new signatures.
% Warning: If you have more than five readers you are out of luck,
% because it will overflow to a new page. You may try to put part of
% the title in with the name.
\reader{First}{Dr. Prathap Ramamurthy}{Associate Professor of Mechanical Engineering, Thesis Advisor}
\reader{Second}{Dr. Feridun Delale}{Chair, Dept. of Mechanical Engineering}

%%%%%%%%%%%%%%%%%%%%%%%%%%%%%%%%%%%%%%%%%%%%%%%%%%%%%%%%%%%%%%%%  

%                       PRELIMINARY PAGES
% According to the BU guide the preliminary pages consist of:
% title, copyright (optional), approval,  acknowledgments (opt.),
% abstract, preface (opt.), Table of contents, List of tables (if
% any), List of illustrations (if any). The \tableofcontents,
% \listoffigures, and \listoftables commands can be used in the
% appropriate places. For other things like preface, do it manually
% with something like \newpage\section*{Preface}.

% This is an additional page to print a boxed-in title, author name and
% degree statement so that they are visible through the opening in BU
% covers used for reports. This makes a nicely bound copy. Uncomment only
% if you are printing a hardcopy for such covers. Leave commented out
% when producing PDF for library submission.
%\buecethesistitleboxpage

% Make the titlepage based on the above information.  If you need
% something special and can't use the standard form, you can specify
% the exact text of the titlepage yourself.  Put it in a titlepage
% environment and leave blank lines where you want vertical space.
% The spaces will be adjusted to fill the entire page.
\maketitle
\cleardoublepage

% Here goes your favorite quote. This page is optional.
\newpage
%\thispagestyle{empty}÷
\phantom{.}
\vspace{4in}



% \vspace{0.7in}
%
% \noindent
% [The descent to Avernus is easy; the gate of Pluto stands open night
% and day; but to retrace one's steps and return to the upper air, that
% is the toil, that the difficulty.]

\cleardoublepage

% The acknowledgment page should go here. Use something like
% \newpage\section*{Acknowledgments} followed by your text.
\newpage
\section*{\centerline{Acknowledgments}}
Here go all your acknowledgments. You know, your advisor, funding agency, lab
mates, etc., and of course your family.

It is difficult to narrow down who to thank for helping me get to this point in my career, but here, I make an attempt to do so.

First, I would like to thank my family. Mom, Dad, and Abuela - you have all been esp. David, etc. etc. Do this section partially in Spanish to honor Mom and Abuela, especially.


Tiffany, etc. etc. Luis et. etc. Richard etc. etc.

Jered and Soobeen, etc. etc. Liz and Dennis, etc. etc.

Haoxiang etc. etc. Prathap etc. etc. Taehun etc. etc. Jimmy and Johnny, etc. etc.

NOAA-CESSRST, etc. etc. Harold Gamarro, etc. etc. Veeshan etc. etc. Omar etc. etc. Rob Defelice etc. etc.

\vskip 1in

\noindent
Gabriel Angel Rios\\
May 25, 2022
\cleardoublepage

% The abstractpage environment sets up everything on the page except
% the text itself.  The title and other header material are put at the
% top of the page, and the supervisors are listed at the bottom.  A
% new page is begun both before and after.  Of course, an abstract may
% be more than one page itself.  If you need more control over the
% format of the page, you can use the abstract environment, which puts
% the word "Abstract" at the beginning and single spaces its text.

\begin{abstractpage}
% ABSTRACT

The atmospheric boundary layer is crucial to the exchange in energy between the Earth's surface and the atmosphere. Within this layer, the majority of human activities are carried out, which makes understanding the boundary layer especially important for many of our interests. A key component of this energy exchange is found at the surface, was surface properties are the interface through which momentum, heat, moisture, and other fluxes are transferred between media. Not only does the surface act as an interface, but as an actor that influences the exchange efficiency and rates. This concept is the crux of atmospheric boundary layer research.

Parallel to activities concentrating at the surface, human activity tends to congregate in cities, with populations becoming increasingly concentrated in urban areas as the 21st century progresses. Within urban areas, heavy and dense populations result in significantly altered land surface properties and introduce human-induced (also known as \textit{anthropogenically-induced}) sources of momentum, heat, moisture, and aerosols. The land surface modifications and anthropogenic fluxes introduced by urban areas has had a significant effect on urban meteorology, the bulk of which has occurred in the boundary layer. These factors contribute to the create a complex thermal and momentum layer with various levels of mixing and sublayers. This phenomenon is referred to as the urban boundary layer (UBL).

The UBL has been extensively investigated in an effort to better understand the physics of UBL processes and their effects on public health and infrastructure resilience. Moreover, research into the UBL is crucial for improving weather forecasts and informing urban planning strategies, both of which are making concerted efforts to adapt to the latest knowledge in this field. However, several gaps still exist in the literature on this topic. Specifically, the effects of urbanization on boundary layer structure and dynamics are not fully understood, especially in the vertical direction. To a degree, this is a result of the inability to observe momentum, heat, and moisture beyond the surface. 

Herein, an improved understanding of the UBL is presented using remote sensing methods to provide new information on the UBL. First, a new method for estimating surface fluxes using satellite data is introduced. Then, a comprehensive study of the climatology of UBL momentum, heat, and moisture over New York City is presented using ground-based profiling methods. Finally, a complementary study to the latter focusing on UBL dynamics and turbulent processes is introduced, which is in progress at the time of the publication of this thesis. In summary, the work presented in this thesis attempts to leverage remote sensing methods to improve our understanding of the relationship between urban areas and the atmosphere to inform the stakeholders that help protect and plan for safeguarding life and property in cities.

\end{abstractpage}
\cleardoublepage

% Now you can include a preface. Again, use something like
% \newpage\section*{Preface} followed by your text

% Table of contents comes after preface
\tableofcontents
\cleardoublepage

% If you do not have tables, comment out the following lines
\newpage
\addcontentsline{toc}{chapter}{\listtablename}
\listoftables
\cleardoublepage

% If you have figures, uncomment the following line
\newpage
\addcontentsline{toc}{chapter}{\listfigurename}
\listoffigures
\cleardoublepage

% List of Abbrevs is NOT optional (Martha Wellman likes all abbrevs listed)
\chapter*{List of abbreviations and symbols}

\begin{center}
  \begin{tabular}{lll}
    \hspace*{2em} & \hspace*{1in} & \hspace*{4.5in} \\
    ASOS  & \dotfill & Automated Surface Observation Station \\
    $C_H$  & \dotfill & Bulk heat transfer coefficient \\
    $h_0$  & \dotfill & Element roughness height \\
    L & \dotfill & Obukhov length \\
    LST  & \dotfill & Land surface temperature \\
    MOST  & \dotfill & Monin-Obukhov similarity theory \\
    NLCD  & \dotfill & National Land Cover Database \\
    NWP  & \dotfill & Numerical weather prediction \\
    PBL  & \dotfill & Planetary boundary layer \\
    PBLH  & \dotfill & Planetary boundary layer height \\
    q & \dotfill & Specific humidity \\
    Re  & \dotfill & Reynolds number \\
    $Q_H$  & \dotfill & Surface sensible heat flux \\
    $Q_L$  & \dotfill & Surface latent heat flux \\
    $Q_S$  & \dotfill & Storage heat flux \\
    $T_air$  & \dotfill & \SI{2}{\meter} air temperature \\
    $u_*$ & \dotfill & Friction velocity \\
    u & \dotfill & Zonal wind component \\
    U & \dotfill & Mean horizontal wind \\
    UBL  & \dotfill & Urban boundary layer \\
    UHI  & \dotfill & Urban heat island \\
    USEB  & \dotfill & Urban surface energy budget \\
    uWRF  & \dotfill & Urbanized Weather Research and Forecasting Model \\
    v & \dotfill & Meridional wind component \\
    w & \dotfill & Vertical wind component \\
    z  & \dotfill & Height \\
     & & \\
    $\kappa$  & \dotfill & von Karman constant \\
    $\theta$  & \dotfill & Potential temperature \\
    $\zeta$  & \dotfill & Atmospheric stability \\
  \end{tabular}
\end{center}
\cleardoublepage

% END OF THE PRELIMINARY PAGES

\newpage
\endofprelim

% ABSTRACT

The atmospheric boundary layer is crucial to the exchange in energy between the Earth's surface and the atmosphere. Within this layer, the majority of human activities are carried out, which makes understanding the boundary layer especially important for many of our interests. A key component of this energy exchange is found at the surface, was surface properties are the interface through which momentum, heat, moisture, and other fluxes are transferred between media. Not only does the surface act as an interface, but as an actor that influences the exchange efficiency and rates. This concept is the crux of atmospheric boundary layer research.

Parallel to activities concentrating at the surface, human activity tends to congregate in cities, with populations becoming increasingly concentrated in urban areas as the 21st century progresses. Within urban areas, heavy and dense populations result in significantly altered land surface properties and introduce human-induced (also known as \textit{anthropogenically-induced}) sources of momentum, heat, moisture, and aerosols. The land surface modifications and anthropogenic fluxes introduced by urban areas has had a significant effect on urban meteorology, the bulk of which has occurred in the boundary layer. These factors contribute to the create a complex thermal and momentum layer with various levels of mixing and sublayers. This phenomenon is referred to as the urban boundary layer (UBL).

The UBL has been extensively investigated in an effort to better understand the physics of UBL processes and their effects on public health and infrastructure resilience. Moreover, research into the UBL is crucial for improving weather forecasts and informing urban planning strategies, both of which are making concerted efforts to adapt to the latest knowledge in this field. However, several gaps still exist in the literature on this topic. Specifically, the effects of urbanization on boundary layer structure and dynamics are not fully understood, especially in the vertical direction. To a degree, this is a result of the inability to observe momentum, heat, and moisture beyond the surface. 

Herein, an improved understanding of the UBL is presented using remote sensing methods to provide new information on the UBL. First, a new method for estimating surface fluxes using satellite data is introduced. Then, a comprehensive study of the climatology of UBL momentum, heat, and moisture over New York City is presented using ground-based profiling methods. Finally, a complementary study to the latter focusing on UBL dynamics and turbulent processes is introduced, which is in progress at the time of the publication of this thesis. In summary, the work presented in this thesis attempts to leverage remote sensing methods to improve our understanding of the relationship between urban areas and the atmosphere to inform the stakeholders that help protect and plan for safeguarding life and property in cities.
